
\documentclass[12pt]{article}

\usepackage{amssymb}
\usepackage{amsmath}
\usepackage[pdftex]{graphicx}
\usepackage[utf8x]{inputenc}
\usepackage[brazil]{babel}
\usepackage{hyperref}
\usepackage{bm}

\usepackage{graphicx}
\usepackage{fancyhdr}
\usepackage{slashed}
\usepackage{setspace}
\linespread{1.5}

\newcommand{\mc}{\mathcal}
\newcommand{\nn}{\nonumber \\ }

\begin{document}

\title{Matching Hla reads}
\author{DM JC VA}
\maketitle 

\section{Probabilistic Model}

Let $\mc{R}= \left\{r_1,\ldots, r_N\right\}$ a set of reads mapped into a set
$\mc T = \left\{t_1,\ldots,t_K\right\}$ of transcripts (or targets).  The
information of which read is mapped into which target can be summarized by
the indicator matrix $\bm X$, where $X_{nk} = 1$ if the read $n$ is mapped on
the target $k$ and $0$.

Let $\bm \theta =(\theta_1, \ldots, \theta_K)$ where $\theta_k$ is 
the probability that one read is generated from a transcript $t_k$, then the
log-likelihood of $\bm \theta$ given $\bm X$ is, 
\begin{align}
    \label{eq:}
    \mc L (\bm \theta; \bm X) = \log P(\bm X| \bm \theta) = \sum_{n=1}^N 
    \log\left(\sum_{k=1}^K X_{nk}\frac{\theta_k}{\tilde l_k}\right)
\end{align}
where $\tilde l_k$ is the effective length of the transcript. This
probabilistic model is understood assuming that the probability of one read 
maps into one transcript depends solely of its expression. This model was 
proposed by \cite{} and \cite{}î and further developed by \cite{}.
The maximum likelihood solution of this model cannot be solved
analytically, nevertheless, it can be accomplished by Estimation Maximization
(EM) procedure. 

Defining the auxiliary and unmeasured matrix $\bm Z$ that is the true indicator
matrix, where $Z_{nk}=1$ only if the read $n$ was originated by the transcript
$k$ and $0$ otherwise. Then, the complete log-likelihood function of the model
is 
\begin{align}
    \label{eq:}
    \mc L (\bm \theta; \bm X, \bm Z) = \log P(\bm X, \bm Z| \bm \theta) = \sum_{n=1}^N 
    \sum_{k=1}^K Z_{nk}\log\left( X_{nk}\frac{\theta_k}{\tilde l_k}\right)
\end{align}

\subsection{EM algorithm}

Given the complete model, it can be show that EM algorithms steps are given by,
\paragraph{Estimation step:}
\begin{align}
    \gamma_{nk}^{(t+1)} &= \mathbb E (Z_{nk}|\bm X_n, \bm \theta^t) 
    = P(Z_{nk} = 1 | \bm X_n, \bm \theta^t) \nn 
    &= \frac{X_{nk}\theta_k^t}{\sum_{j=1}^KX_{nj}\theta_j^t}
    \label{eq:E}
\end{align}
\paragraph{Maximisation step:} Let $n_k^{(t+1)} = \sum_{n=1}^N
\gamma_{nk}^{(t+1)}$ then 
\begin{align}
    \theta_k^{(t+1)} = \frac{n_k^{(t+1)}}{N}
    \label{eq:M}
\end{align}

\subsection{Implementation}

From equation EM equations it follows,
\begin{align}
    \label{eq:}
    n_k^{(t+1)} &= \sum_{n=1}^N \gamma_{nk}^{(t+1)} 
    = \sum_{n=1}^N\frac{X_{nk}\theta_k^t}{\sum_{j=1}^KX_{nj}\theta_j^t}\nn
    &= n_{u(k)} 
    + \sum_{j\neq k} \frac{\theta_k^t}{\theta_k^t + \theta_j^t}n_{u(k,j)}
    + \sum_{l\neq j}\sum_{j\neq k} 
    \frac{\theta_k^t}{\theta_k^t + \theta_j^t+ \theta_l^t} n_{u(k,j,l)} +
    \ldots
\end{align}
Where $n_{u(k)}$ is the number of reads that maps uniquely to
the targets, $t_k$ and $n_{u(k,j)}$ and uniquely mapped in $t_k$ and $t_j$,
etc. Since any initial condition will lead to the same global maximum we
set $\theta_k^0 = n_{u(k)}/N$.
%\bibliographystyle{plain}
%\bibliography{biblio}
\end{document}
